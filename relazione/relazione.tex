\documentclass[12pt,italian]{article}
\usepackage[italian]{babel}
\usepackage[a4paper, margin=1.97cm]{geometry}
\usepackage{graphicx}
\usepackage{amsmath}
\usepackage{caption}
\usepackage{subcaption}
\usepackage{amsfonts}
\usepackage{cleveref}
\usepackage{booktabs}
\usepackage{xcolor}
\graphicspath{{./images/}}

\newcommand{\err}[1]{\textcolor{red}{#1}}
\newcommand{\ext}[1]{\textcolor{blue}{#1}}
\crefname{table}{tab.}{tab.}

\title{Percolazione e robustezza nelle reti complesse: analisi della transizione di fase nel modello Erdős-Rényi-Gilbert}
\author{Giacomo Cicala}
\date{\today}

\begin{document}
\maketitle

\renewcommand{\abstractname}{Abstract}
\begin{abstract}
In questo lavoro, esploriamo il fenomeno della percolazione e della robustezza nelle reti complesse, concentrandoci in particolare sulla transizione di fase che si verifica nel modello di Erdős-Rényi-Gilbert. Attraverso l'analisi teorica e la simulazione numerica, dimostriamo come la connettività di una rete possa subire un cambiamento drastico al superamento di una soglia critica, evidenziando le implicazioni di questo fenomeno per la resilienza delle reti reali.
\end{abstract}

\section{Introduzione}
Esiste un'ampia varietà di sistemi complessi che emergono in natura, nella società e nella tecnologia aventi delle strutture a network. Questi sistemi sono caratterizzati da un gran numero di componenti interagenti, che danno origine a comportamenti collettivi emergenti che non possono essere facilmente previsti o compresi partendo dalla sola conoscenza dei singoli componenti. \ext{ Ad esempio, la cellula è descritta efficacemente come una rete complessa di sostanze chimiche connesse da reazioni chimiche; Internet è una rete complessa di router e computer collegati tramite vari collegamenti fisici o wireless; mode e idee si diffondono sulle reti sociali, dove i nodi sono esseri umani e i link rappresentano diverse relazioni sociali.} La grande scoperta della network science è che, nonostante l'evidente diversità dei sistemi complessi, la struttura e l'evoluzione delle reti sottostanti a ciascun sistema sono guidate da un insieme comune di leggi e principi fondamentali. Pertanto, nonostante le sorprendenti differenze in termini di forma, dimensione, natura e scopo delle reti reali, la maggior parte di esse è governata da principi organizzativi comuni. Una volta astratta la natura dei componenti e la specifica tipologia delle interazioni tra di essi, le reti risultanti appaiono più simili che differenti tra loro \cite[Sez. 1.2-1.3]{barabasi2016network}. \ext{ Di conseguenza, possiamo utilizzare un insieme comune di strumenti matematici per esplorare questi sistemi.} 

La Network Science si pone l'obiettivo di costruire modelli in grado di riprodurre le proprietà delle reti reali.
Dal punto di vista della modellistica, una rete è un oggetto relativamente semplice, costituito soltanto da nodi e link. La vera sfida, tuttavia, risiede nel decidere come posizionare i link tra i nodi affinché si possa riprodurre la complessità di un sistema reale. Tuttavia, la maggior parte delle reti osservabili non possiede una struttura regolare evidente; al contrario, a una prima analisi esse appaiono come se si fossero organizzate in modo casuale.
A tal proposito, la filosofia alla base dei Random Network è semplice: si assume che questo obiettivo venga raggiunto al meglio distribuendo i link tra i nodi in modo puramente casuale . Il modello di Random Network più ampiamente studiato è il modello di Erdős-Rényi-Gilbert, che verrà approfondito nelle sezioni successive \cite[Sez. 3.2]{barabasi2016network}.

\section{Elementi di teoria dei grafi}
Prima di addentrarci nell'analisi del modello di Erdős-Rényi-Gilbert, è necessario introdurre alcuni concetti fondamentali della teoria dei grafi \cite[Sez. 2.3, 2.10]{barabasi2016network}. Un grafo $G$ è costituito da un insieme di nodi e un insieme di link che connettono coppie di nodi, rappresentando un'interazione tra di essi. La dimensione del grafo è data dal numero di nodi $N$, mentre la connettività è data dal numero totale di link $L$.
Una proprietà fondamentale di ciascun nodo è il suo grado, che rappresenta il numero di link che esso possiede verso altri nodi.
In un grafo non orientato, il numero totale di link $L$, può essere espresso come \ext{la somma dei gradi dei nodi divisa per 2, poiché ogni link viene contato due volte (una volta per ciascuno dei due nodi che collega)}:
\begin{equation}
    L = \frac{1}{2} \sum_{i=1}^{N} k_i
\end{equation}
dove $k_i$ è il grado del nodo $i$. Il grado medio $\langle k \rangle$ è dato da:
\begin{equation}
    \langle k \rangle = \frac{1}{N} \sum_{i=1}^{N} k_i = \frac{2L}{N}
\end{equation}
La distribuzione dei gradi, $P(k)$, fornisce la probabilità che un nodo scelto casualmente nella rete abbia grado $k$. Poiché $P(k)$ è una probabilità, deve essere normalizzata, cioè:
\begin{equation}
    \sum_{k=1}^{\infty} P(k) = 1
\end{equation}
Per una rete con $N$ nodi, la distribuzione dei gradi è data da:
\begin{equation}
    P(k) = \frac{N_k}{N}
\end{equation}
dove $N_k$ è il numero di nodi di grado $k$.

Il coefficiente di clustering descrive il grado con cui i vicini di un dato nodo sono collegati tra loro. Per un nodo $i$ con grado $k_i$, il coefficiente di clustering locale è definito come:
\begin{equation}
    C_i = \frac{2 L_i}{k_i (k_i - 1)}
\end{equation}
dove $L_i$ rappresenta il numero di link tra i $k_i$ vicini del nodo $i$. Si noti che $C_i$ è compreso tra 0 e 1. \ext{Dunque,
$C_i$ misura la densità locale dei link nel network: più densamente interconnesso è il "vicinato" del nodo $i$, maggiore è il suo coefficiente di clustering locale.}
Il grado di clustering di un intero network è catturato dal coefficiente di clustering medio $\langle C \rangle$,definito come:
\begin{equation}
    \langle C \rangle = \frac{1}{N} \sum_{i=1}^{N} C_i
\end{equation}
In linea con l'interpretazione probabilistica, 〈C〉 è la probabilità che due vicini di un nodo scelto casualmente siano collegati tra loro.

\section{Modello di Erdős-Rényi-Gilbert}
\section{Simulazione e risultati}
\section{Conclusioni}

\bibliographystyle{plain}
\bibliography{bibliografia}

\err{AGGIUSTA BIBLIO (pagine si/no?), ecc}

\end{document}