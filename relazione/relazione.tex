\documentclass[12pt,italian]{article}
\usepackage[italian]{babel}
\usepackage[a4paper, margin=1.97cm]{geometry}
\usepackage{graphicx}
\usepackage{amsmath}
\usepackage{caption}
\usepackage{subcaption}
\usepackage{amsfonts}
\usepackage{cleveref}
\usepackage{booktabs}
\usepackage{xcolor}
\graphicspath{{./images/}}

\newcommand{\err}[1]{\textcolor{red}{#1}}
\crefname{table}{tab.}{tab.}

\title{Transizione di fase nel modello Erdős-Rényi-Gilbert: l'emersione della componente gigante}
\author{Giacomo Cicala}
\date{\today}

\begin{document}
\maketitle

\renewcommand{\abstractname}{Abstract}
\begin{abstract}
    
\end{abstract}

\section{Introduzione}

\err{Aggiungi esempi di reti reali: sociali, internet, biologiche, ecc.}

These systems are collectively called complex systems, capturing the fact that it is difficult to derive their collective behavior from a knowledge of the system’s components. Given the important role complex systems play in our daily life, in science and in economy, their understanding, mathematical description, prediction, and eventually control is one of the major intellectual and scientific challenges of the 21st century.
At the end, networks permeate science, technology, business and nature to a much higher degree than it may be evident upon a casual inspection. Consequently, we will never understand complex systems unless we develop a deep understanding of the networks behind them.

The exploding interest in network science during the first decade of the 21st century is rooted in the discovery that despite the obvious diversity of complex systems, the structure and the evolution of the networks behind each system is driven by a common set of fundamental laws and principles. Therefore, notwithstanding the amazing differences in form, size, nature, age, and scope of real networks, most networks are driven by common organizing principles. Once we disregard the nature of the components and the precise nature of the interactions between them, the obtained networks are more similar than different from each other.\cite{albert2002statistical}\cite{barabasi2016network}

A key discovery of network science is that the architecture of networks emerging in various domains of science, nature, and technology are similar to each other, a consequence of being governed by the same organizing principles. Consequently we can use a common set of mathematical tools to explore these systems.
suca
\bibliographystyle{plain}
\bibliography{bibliografia}

\err{AGGIUSTA BIBLIO (pagine si/no?), ecc}

\end{document}