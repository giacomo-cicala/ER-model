\documentclass[12pt,italian]{article}
\usepackage[italian]{babel}
\usepackage[a4paper, margin=1.97cm]{geometry}
\usepackage{graphicx}
\usepackage{amsmath}
\usepackage{caption}
\usepackage{subcaption}
\usepackage{amsfonts}
\usepackage{cleveref}
\usepackage{booktabs}
\usepackage{xcolor}
\graphicspath{{./images/}}

\newcommand{\err}[1]{\textcolor{red}{#1}}
\newcommand{\ext}[1]{\textcolor{blue}{#1}}
\crefname{table}{tab.}{tab.}

\title{Transizione di fase nel modello Erdős-Rényi-Gilbert: l'emersione della componente gigante}
\author{Giacomo Cicala}
\date{\today}

\begin{document}
\maketitle

\renewcommand{\abstractname}{Abstract}
\begin{abstract}
    Il presente elaborato analizza le proprietà strutturali dei grafi casuali secondo il modello di Erdős-Rényi $G(N, p)$, con particolare attenzione ai fenomeni critici che emergono al variare della connettività media.
Attraverso un approccio teorico di campo medio (Mean Field), basato sulla teoria dei processi di ramificazione, viene derivata analiticamente la soglia di percolazione $\langle k \rangle_c = 1$ e l'equazione di autoconsistenza per la dimensione della componente gigante ($S$).
Le previsioni teoriche sono state verificate mediante simulazioni numeriche su reticoli di dimensione finita. I risultati confermano l'esistenza di una transizione di fase topologica del secondo ordine: in corrispondenza del punto critico, si osserva l'emersione spontanea dell'ordine (rottura di simmetria) e la divergenza della dimensione media dei cluster finiti, analoga alla suscettività nei sistemi ferromagnetici descritti dalla teoria di Landau.
\end{abstract}

\section{Introduzione}
Esiste un'ampia varietà di sistemi complessi che emergono in natura, nella società e nella tecnologia aventi delle strutture a network. Questi sistemi sono caratterizzati da un gran numero di componenti interagenti, che danno origine a comportamenti collettivi emergenti che non possono essere facilmente previsti o compresi partendo dalla sola conoscenza dei singoli componenti. \ext{ Ad esempio, la cellula è descritta efficacemente come una rete complessa di sostanze chimiche connesse da reazioni chimiche; Internet è una rete complessa di router e computer collegati tramite vari collegamenti fisici o wireless; mode e idee si diffondono sulle reti sociali, dove i nodi sono esseri umani e i link rappresentano diverse relazioni sociali.} La grande scoperta della network science è che, nonostante l'evidente diversità dei sistemi complessi, la struttura e l'evoluzione delle reti sottostanti a ciascun sistema sono guidate da un insieme comune di leggi e principi fondamentali. Pertanto, nonostante le sorprendenti differenze in termini di forma, dimensione, natura e scopo delle reti reali, la maggior parte di esse è governata da principi organizzativi comuni. Una volta astratta la natura dei componenti e la specifica tipologia delle interazioni tra di essi, le reti risultanti appaiono più simili che differenti tra loro \cite[Sez. 1.2-1.3]{barabasi2016network}. \ext{ Di conseguenza, possiamo utilizzare un insieme comune di strumenti matematici per esplorare questi sistemi.} 

La Network Science si pone l'obiettivo di costruire modelli in grado di riprodurre le proprietà delle reti reali.
Dal punto di vista della modellistica, una rete è un oggetto relativamente semplice, costituito soltanto da nodi e link. La vera sfida, tuttavia, risiede nel decidere come posizionare i link tra i nodi affinché si possa riprodurre la complessità di un sistema reale. Tuttavia, la maggior parte delle reti osservabili non possiede una struttura regolare evidente; al contrario, a una prima analisi esse appaiono come se si fossero organizzate in modo casuale.
A tal proposito, la filosofia alla base dei Random Network è semplice: si assume che questo obiettivo venga raggiunto al meglio distribuendo i link tra i nodi in modo puramente casuale . Il modello di Random Network più ampiamente studiato è il modello di Erdős-Rényi-Gilbert, che approfondiremo nelle sezioni successive \cite[Sez. 3.2]{barabasi2016network}.

\section{Elementi di teoria dei grafi}
\section{Modello di Erdős-Rényi-Gilbert}
\section{Simulazione e risultati}
\section{Conclusioni}

\bibliographystyle{plain}
\bibliography{bibliografia}

\err{AGGIUSTA BIBLIO (pagine si/no?), ecc}

\end{document}