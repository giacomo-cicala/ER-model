\documentclass[12pt,italian]{article}
\usepackage[italian]{babel}
\usepackage[a4paper, margin=1.97cm]{geometry}
\usepackage{graphicx}
\usepackage{amsmath}
\usepackage{caption}
\usepackage{subcaption}
\usepackage{amsfonts}
\usepackage{booktabs}
\usepackage[dvipsnames]{xcolor}
\usepackage[
    colorlinks=true,
    linkcolor=blue,
    citecolor=blue,
    urlcolor=blue
]{hyperref}
\usepackage{cleveref}
\usepackage{csquotes}
\usepackage[style=phys, backend=biber, biblabel=brackets]{biblatex}

\addbibresource{bibliografia.bib}
\DeclareFieldFormat{postnote}{\textcolor{blue}{\mkpageprefix[pagination]{#1}}}
\DeclareFieldFormat{multipostnote}{\textcolor{blue}
{\mkpageprefix[pagination]{#1}}}
\DeclareFieldFormat{cite}{\hyperref{\mkbibbrackets{#1}}}
\renewcommand{\mkbibbrackets}[1]{\textcolor{blue}{[#1]}}

\crefformat{equation}{#2(#1)#3}
\crefformat{appendix}{#2(#1)#3}
\graphicspath{{./images/}}

\newcommand{\err}[1]{\textcolor{red}{#1}}
\newcommand{\ext}[1]{\textcolor{blue}{#1}}
\crefname{table}{tab.}{tab.}

\title{Percolazione e robustezza nelle reti complesse: analisi della transizione di fase nel modello Erdős-Rényi-Gilbert}
\author{Giacomo Cicala}
\date{\today}

\begin{document}
\maketitle

\renewcommand{\abstractname}{Abstract}

\section{Introduzione}
\err{SISTEMA}
Negli ultimi decenni, lo studio dei sistemi complessi ha subito una profonda accelerazione grazie allo sviluppo della \emph{Network Science}, un framework teorico che permette di descrivere sistemi apparentemente disparati, dalle reti metaboliche nella cellula, alle infrastrutture tecnologiche come Internet, fino alle reti sociali, attraverso un linguaggio matematico unificato \cite[Cap. 1]{barabasi2016network}.
Una delle questioni centrali in questo ambito riguarda l'emergere spontaneo di proprietà macroscopiche a partire da interazioni locali disordinate. In particolare, è fondamentale comprendere i meccanismi che garantiscono la connettività globale del sistema e, di riflesso, la sua \textbf{robustezza} a fronte di fallimenti casuali dei componenti (\emph{random failures}).

Il presente elaborato si propone di analizzare la stabilità strutturale delle reti attraverso lo studio della \emph{teoria della percolazione}, applicata al modello di grafi casuali di Erdős-Rényi $G(N, p)$.
Come discusso da Newman \cite[Cap. 16]{newman2018networks}, la percolazione rappresenta una delle transizioni di fase geometriche più semplici e fondamentali: essa descrive il passaggio improvviso da uno stato frammentato, in cui il sistema è composto da isole disconnesse, a uno stato connesso dominato da una \emph{componente gigante} che abbraccia l'intero sistema.

Sebbene le reti reali presentino spesso correlazioni topologiche complesse (come la proprietà \emph{scale-free}), il modello di Erdős-Rényi-Gilbert costituisce il "modello nullo" di riferimento, analogo al gas ideale in termodinamica. In questo modello, la probabilità di connessione tra i nodi è uniforme e indipendente, permettendo di trattare la transizione di fase in approssimazione di campo medio (\emph{Mean Field}).
L'analisi di questo sistema possiede una forte valenza fisica: la soglia critica di percolazione identifica il punto esatto di rottura del sistema. Sotto questa soglia, la rete perde la sua funzionalità globale; sopra di essa, emerge l'ordine a lungo raggio.

\subsection{Obiettivi del lavoro}
\err{SISTEMA}
Lo scopo centrale di questo elaborato è lo studio della \emph{teoria della percolazione} come chiave di lettura fondamentale per comprendere le transizioni di fase topologiche nelle reti complesse.
Utilizzando il modello di Erdős-Rényi-Gilbert come sistema di riferimento, il lavoro si prefigge i seguenti obiettivi:

\begin{itemize}
    \item \textbf{Analisi della soglia di percolazione:} Derivare analiticamente, mediante il formalismo delle funzioni generatrici \cite[Cap. 12]{newman2018networks}, la condizione critica $\langle k \rangle_c = 1$. Questo valore segna l'emersione del \emph{cluster di percolazione} (o componente gigante), corrispondente alla rottura di simmetria del sistema e alla nascita dell'ordine a lungo raggio.

    \item \textbf{Caratterizzazione dei fenomeni critici:} Studiare il comportamento della \emph{suscettività topologica} ($\chi$), intesa come la dimensione media dei cluster finiti. Attraverso simulazioni numeriche, si intende verificare la divergenza di $\chi$ in prossimità della soglia critica, evidenziando la natura continua (del secondo ordine) della transizione, in analogia con i sistemi termodinamici vicino al punto critico.

    \item \textbf{Implicazioni sulla robustezza:} Interpretare fisicamente la transizione di percolazione nel contesto della stabilità strutturale. Si discuterà come la scomparsa del cluster di percolazione (per $\langle k \rangle < 1$) corrisponda al punto di collasso funzionale (\emph{failure}) della rete, collegando così la meccanica statistica della percolazione al concetto di robustezza descritto da Barabási \cite[Cap. 8]{barabasi2016network}.
\end{itemize}

\section{Elementi di teoria dei grafi}
Prima di addentrarci nell'analisi del modello di Erdős-Rényi-Gilbert, è necessario introdurre alcuni concetti fondamentali della teoria dei grafi \cite[Sez. 2.3, 2.10]{barabasi2016network}. Un grafo $G$ è costituito da un insieme di nodi e un insieme di link che connettono coppie di nodi, rappresentando un'interazione tra di essi. La dimensione del grafo è data dal numero di nodi $N$, mentre la connettività è data dal numero totale di link $L$.
Una proprietà fondamentale di ciascun nodo è il suo grado $k$, che rappresenta il numero di link che esso possiede verso altri nodi.
In un grafo non orientato, il numero totale di link $L$, può essere espresso come \ext{la somma dei gradi dei nodi divisa per 2, poiché ogni link viene contato due volte (una volta per ciascuno dei due nodi che collega)}:
\begin{equation}
    L = \frac{1}{2} \sum_{i=1}^{N} k_i
\end{equation}
dove $k_i$ è il grado del nodo $i$. Il grado medio $\langle k \rangle$ è dato da:
\begin{equation}
    \langle k \rangle = \frac{1}{N} \sum_{i=1}^{N} k_i = \frac{2L}{N}
    \label{eq:grado_medio}
\end{equation}
La distribuzione dei gradi, $P(k)$, fornisce la probabilità che un nodo scelto casualmente nella rete abbia grado $k$. Poiché $P(k)$ è una probabilità, deve essere normalizzata, cioè:
\begin{equation}
    \sum_{k=1}^{\infty} P(k) = 1
\end{equation}
Per una rete con $N$ nodi, la distribuzione dei gradi è data da:
\begin{equation}
    P(k) = \frac{N_k}{N}
\end{equation}
dove $N_k$ è il numero di nodi di grado $k$.
\section{Modello di Erdős-Rényi-Gilbert}
Il modello di Erdős-Rényi-Gilbert, denotato come $G(N, p)$, è un modello di grafo casuale in cui ogni coppia di $N$ nodi è connessa da un link con probabilità $p$.
In questo modello, il numero totale di link $L$ è una variabile casuale che segue una distribuzione binomiale
\begin{equation}
    P(L) = \binom{\binom{N}{2}}{L} p^L (1-p)^{\binom{N}{2} - L}
\end{equation}
con media
\begin{equation}
    \langle L \rangle = p \binom{N}{2}
\end{equation}
Di conseguenza, utilizzando \cref{eq:grado_medio} il grado medio $\langle k \rangle$ è dato da
\begin{equation}
    \langle k \rangle = \frac{2 \langle L \rangle}{N} = p (N - 1)
\end{equation}
Anche la distribuzione dei gradi $P(k)$ risulta essere binomiale:
\begin{equation}
    P(k) = \binom{N-1}{k} p^k (1-p)^{N-1-k}
\end{equation}
In molti casi siamo interessati a grandi network con grado medio finito $\langle k \rangle$ indipendente dalla dimensione $N$. In questo limite, ovvero $ \langle k \rangle \ll N $, la distribuzione dei gradi si approssima a una distribuzione di Poisson:
\begin{equation}
    P(k) = \frac{\langle k \rangle^k e^{-\langle k \rangle}}{k!}
\end{equation}

\section{Transizione di fase di percolazione}
\subsection{Emergenza della componente gigante}
\err{SISTEMA}
Consideriamo il caso in cui $p = 0$: non vi sono link nella rete ed essa è completamente sconnessa. \ext{ Ogni vertice è un'isola a sé stante; la rete possiede $N$ componenti separate costituite esattamente da un singolo vertice ciascuna}. Nel limite opposto, quando $p = 1$, ogni vertice è connesso direttamente a tutti gli altri, creando un'unica componente che copre l'intera rete. Concentriamoci ora sulla dimensione della componente più grande della rete in ciascuno di questi casi. Nel primo caso ($p = 0$) la componente più grande ha dimensione 1. Nel secondo ($p = 1$) la componente più grande ha dimensione $N$. Nel primo caso, la dimensione della componente più grande è indipendente dal numero di vertici $N$ della rete; nel secondo è proporzionale a $N$, ovvero è estensiva. \ext{ In molte applicazioni delle reti è cruciale che esista una componente che riempia la gran parte della rete. Per esempio, nel caso di Internet, è importante che vi sia un cammino attraverso la rete che colleghi la maggior parte dei computer alla maggior parte degli altri. Se così non fosse, la rete non sarebbe in grado di svolgere la sua funzione prevista di garantire le comunicazioni computer-computer per i suoi utenti}. Una domanda interessante da porsi è come avvenga la transizione tra questi due estremi se costruiamo grafi casuali con valori di $p$ gradualmente crescenti, partendo da 0 e finendo a 1. Potremmo intuire, per esempio, che la dimensione della componente più grande aumenti in qualche modo gradualmente con $p$, diventando estensiva solo nel limite in cui $p = 1$. In realtà, accade qualcosa di molto più interessante. Come vedremo, la dimensione della componente più grande subisce un cambiamento improvviso, o transizione di fase, da dimensione costante a dimensione estensiva in corrispondenza di un particolare valore speciale di $p$. Una componente della rete la cui dimensione cresce proporzionalmente a $n$ è chiamata componente gigante. La dimensione della componente gigante $S$, definita come la frazione di nodi che vi appartengono, nel limite di Poisson è data dall'equazione trascendente:
\begin{equation}  
      \quad \boxed{S = 1 - e^{-\langle k \rangle S}}
       \label{eq:giant_comp}
\end{equation}
e il punto critico di percolazione, ovvero il punto in cui la componente gigante emerge, è dato da
\begin{equation}
    \langle k \rangle_c = 1
\end{equation}
come ricavato in \cref{sec:GC}.

\subsection{Regimi ELIMINARE?}
\ext{
Possiamo distinguere quattro regimi topologici distinti \cite[Cap. 3.6]{barabasi2016network}:}
\begin{itemize}
    \item \textbf{Regime subcritico ($\langle k \rangle < 1$):} In questo regime, la rete è composta da numerosi cluster finiti di dimensione limitata. La dimensione massima dei cluster cresce al massimo come $\log N$, e non esiste una componente gigante. La rete è frammentata e non connessa.

    \item \textbf{Punto critico ($\langle k \rangle = 1$):} In questo punto, la rete subisce una transizione di fase. La dimensione massima dei cluster cresce come $N^{2/3}$, e la suscettività topologica diverge. La rete è ancora frammentata, ma si avvicina alla formazione di una componente gigante.

    \item \textbf{Regime supercritico ($\langle k \rangle > 1$):} In questo regime, emerge una componente gigante che contiene una frazione finita $S$ dei nodi. I cluster finiti rimanenti hanno dimensione limitata. La rete è connessa a livello macroscopico.

    \item \textbf{Regime completamente connesso ($\langle k \rangle > \ln N$):} In questo regime, il numero di aspettazione di nodi isolati è inferiore a 1 \cref{sec:connected_regime}, e la rete è quasi sicuramente completamente connessa.
\end{itemize}
\subsection{Transizione di fase}
Per quantificare la transizione di fase possiamo introduere due quantità fondamentali:
\begin{itemize}
    \item \textbf{Parametro d'ordine $P_\infty$:} Definito come la probabilità che un nodo appartenga alla componente gigante, coincide dunque con la frazione di nodi appartenenti alla componente gigante $S$. Questo parametro permette di distinguere una fase ordinata ($P_\infty > 0$) da una disordinata ($P_\infty = 0$). Per la teoria della percolazione si ha che \err{REF?}:
\begin{equation}
    S \sim (\langle k \rangle - \langle k \rangle_c)^\beta \quad \text{per } \langle k \rangle \to \langle k \rangle_c^+
\end{equation}
con  $\beta > 0$ esponente critico. \err{Ovvero la dimensione della componente gigante cresce in modo continuo partendo da $S=0$ al punto critico.}
    \item \textbf{Suscettività $\chi$:} Definita come la dimensione media dei cluster piccoli $ \langle s \rangle$, rappresenta la sensibilità della rete a variazioni di $\langle k \rangle$. Per la teoria della percolazione si ha che \err{REF?}:
\begin{equation}    \chi \sim |\langle k \rangle - \langle k \rangle_c|^{-\gamma} \quad \text{per } \langle k \rangle \to \langle k \rangle_c
\end{equation}
con $\gamma > 0$ esponente critico. Ovvero la dimensione media dei cluster piccoli diverge al punto critico.
\end{itemize}
Per quanto riguarda il parametro d'ordine, utilizzando \eqref{eq:giant_comp} possiamo espandere l'esponenziale in serie di Taylor al secondo ordine attorno a $\langle k \rangle_c = 1$ e ottenere \err{REF?}:
\begin{equation}
    S \approx 2(\langle k \rangle - 1)
\end{equation}
Quindi, vicino al punto critico, il parametro d'ordine cresce linearmente con $\langle k \rangle - 1$, indicando che l'esponente critico $\beta$ è pari a 1:
\begin{equation}
    S \sim (\langle k \rangle - 1)^1 \quad \text{per } \langle k \rangle \to 1^+
\end{equation}
 \err{Mettere grafico S-K teorico con considerazioni?}

Invece, per quanto riguarda la suscettività, nel regime subcritico la dimensione media delle componenti piccole $ \langle s \rangle$, è data da \cref{sec:small_components}:
\begin{equation}
    \langle s \rangle = \frac{\sum_{s} s^2 N_s}{\sum_{s} s N_s}
\end{equation}
dove $N_s$ è il numero di cluster di dimensione $s$. Si può dimostrare che $\langle s \rangle$ diverge al punto critico con un esponente critico $\gamma = 1$ \cite[Cap. 12.6]{newman2018networks}:
\begin{equation}
    \langle s \rangle \sim \frac{1}{|1 - \langle k \rangle|} 
\end{equation}
indicando la presenza di una transizione di fase continua (del secondo ordine) in cui la dimensione media dei cluster finiti cresce senza limiti al punto critico. Questo comportamento è analogo alla divergenza della suscettività nei sistemi termodinamici vicino al punto critico, rafforzando l'analogia tra la percolazione e le transizioni di fase in fisica statistica. Inoltre, la divergenza di $\langle s \rangle$ al punto critico riflette l'emergere di cluster di tutte le dimensioni secondo una legge a potenza indicando così un'esplosione delle fluttuazioni, caratteristica universale delle transizioni di fase continue.

\err{Mettere grafico divergenza suscettività?}

\section{Percolazione inversa e robustezza}
Una delle caratteristiche di maggiore interesse per un network è la sua \textbf{robustezza}, ovvero la capacità di mantenere la connettività globale nonostante la rimozione di nodi o link. La teoria della percolazione fornisce un quadro teorico per analizzare questo aspetto attraverso il concetto di \emph{percolazione inversa}, in cui si rimuovono nodi o link in modo casuale o mirato, e si studia come ciò influenzi la dimensione della componente gigante.

Analizziamo il caso della rimozione casuale di una frazione $f$ di nodi. Utilizzando il criterio di Molloy-Reed \err{REF?}:
\begin{equation}
    \frac{\langle k^2 \rangle}{\langle k \rangle} > 2
\end{equation}
il quale mi determina la condizione necessaria affinchè esista una componente gigante per un generico network con distribuzione dei gradi $P(k)$, è possibile derivare la soglia critica di percolazione $f_c$ \err{REF?}:
\begin{equation}
    f_c = 1 - \frac{1}{\frac{\langle k^2 \rangle}{\langle k \rangle} - 1}
\end{equation}
oltre la quale la rete collassa e perde la componente gigante. Nel caso del modello di Erdős-Rényi-Gilbert, dove $\langle k^2 \rangle = \langle k \rangle^2 + \langle k \rangle$ \err{REF?}, la soglia critica di percolazione è:
\begin{equation}
    f_c = 1 - \frac{1}{\langle k \rangle}
\end{equation}
Questo risultato mostra che la robustezza di questo network dipende unicamente dal grado medio $\langle k \rangle$: reti con grado medio più elevato sono più robuste alla rimozione casuale di nodi, poiché la soglia critica $f_c$ è più alta. Tuttavia, è sufficiente rimuovere una frazione $f_c < f < 1$ di nodi per far collassare la rete, evidenziando la vulnerabilità intrinseca del modello di Erdős-Rényi-Gilbert a fallimenti casuali. \ext{Questo comportamento è in contrasto con reti con distribuzioni dei gradi più eterogenee (come le reti scale-free), che possono essere molto più robuste a fallimenti casuali ma estremamente vulnerabili a attacchi mirati contro i nodi più connessi \cite[Cap. 8]{barabasi2016network}}.

\section{Simulazione e risultati}
\section{Conclusioni}
\appendix
\section{Appendici}
\subsection{Componente Gigante}\label{sec:GC}
Sia $u$ la frazione di nodi che non appartengono alla componente gigante (GC) e $S$ la frazione di nodi che vi appartengono. Vale la relazione:
\begin{equation}
    u = 1 - S
\end{equation}
Un nodo $i$ non appartiene alla GC se, per ogni altro nodo $j$, o non esiste un arco ($1-p$) oppure esiste un arco ma $j$ non è nella GC ($pu$). La probabilità totale per un singolo nodo $j$ è
   $1 - p + pu$.
Considerando tutti gli $N-1$ possibili vicini, la probabilità che $i$ non sia connesso alla GC è:
\begin{equation}
    u = (1 - p + pu)^{N-1}
\end{equation}
Sostituendo la probabilità di connessione $p = \frac{\langle k \rangle}{N-1}$, passando ai logaritmi e utilizzando l'approssimazione $\ln(1+x) \approx x$ per $x \ll 1$, otteniamo:
\begin{equation}
\ln u = -\langle k \rangle (1-u)
\end{equation}
Esponenziando entrambi i membri otteniamo un'equazione trascendente per $u$:
\begin{equation}
    u = e^{-\langle k \rangle (1-u)}
\end{equation}
Sostituendo $u = 1 - S$, otteniamo l'equazione finale per la dimensione della GC:
\begin{equation}
    S = 1 - e^{-\langle k \rangle S}
\end{equation}
Per trovare il punto critico, cerchiamo quando la derivata del membro di destra rispetto a $S$ valutata in $S=0$ supera la pendenza della retta $y=S$ (che è 1):
\begin{equation}
    \left. \frac{d}{dS} (1 - e^{-\langle k \rangle S}) \right|_{S=0} = 1
\end{equation}
\begin{equation}
    \left. \langle k \rangle e^{-\langle k \rangle S} \right|_{S=0} = 1 \quad
\end{equation}
La transizione di fase avviene dunque a:
\begin{equation}
    \langle k \rangle_c = 1
\end{equation}
\subsection{Regime completamente connesso}\label{sec:connected_regime}
Cerchiamo il valore di $\langle k \rangle$ affinché la rete diventi completamente connessa.
La probabilità che un nodo scelto a caso non abbia link verso la componente gigante è $(1-p)^N$.
Sfruttando l'approssimazione $(1-x/n)^n \approx e^{-x}$, il numero atteso di nodi isolati $I_N$ è:
\begin{equation}
    I_N = N(1-p)^N \approx N e^{-pN} = N e^{-\langle k \rangle}
\end{equation}
La condizione di connettività completa si raggiunge quando rimane al massimo un solo nodo isolato nel sistema ($I_N = 1$). Risolvendo l'equazione:
\begin{equation}
1 = N e^{-\langle k \rangle} \quad \Longrightarrow \quad \langle k \rangle = \ln N
\end{equation}

\subsection{Dimensione media delle componenti piccole} \label{sec:small_components}
Sia $N_s$ il numero di cluster di dimensione $s$ presenti nella rete. La probabilità $\pi_s$ che un nodo scelto a caso appartenga a un cluster di dimensione $s$ è:
\begin{equation}
    \pi_s = \frac{\text{nodi in cluster di taglia } s}{\text{nodi totali}} = \frac{s \cdot N_s}{\sum_{k} k \cdot N_k}
\end{equation}
La dimensione media $\langle s \rangle_{w}$ è quindi il valore atteso della dimensione calcolato su questa distribuzione di probabilità:

\begin{equation}
    \chi = \sum_{s} s \cdot \pi_s = \sum_{s} s \cdot \left( \frac{s N_s}{\sum_{k} k N_k} \right) = \frac{\sum_{s} s^2 N_s}{\sum_{s} s N_s}
\end{equation}

\printbibliography[heading=bibintoc, title={Bibliografia}]

\err{AGGIUSTA BIBLIO (pagine si/no?), ecc}

\end{document}