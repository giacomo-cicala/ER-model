\documentclass[12pt,italian]{article}
\usepackage[italian]{babel}
\usepackage[a4paper, margin=1.97cm]{geometry}
\usepackage{graphicx}
\usepackage{amsmath}
\usepackage{caption}
\usepackage{subcaption}
\usepackage{amsfonts}
\usepackage{cleveref}
\usepackage{booktabs}
\usepackage{xcolor}
\graphicspath{{./images/}}

\newcommand{\err}[1]{\textcolor{red}{#1}}
\newcommand{\ext}[1]{\textcolor{blue}{#1}}
\crefname{table}{tab.}{tab.}

\title{Percolazione e robustezza nelle reti complesse: analisi della transizione di fase nel modello Erdős-Rényi-Gilbert}
\author{Giacomo Cicala}
\date{\today}

\begin{document}
\maketitle

\renewcommand{\abstractname}{Abstract}

\section{Introduzione}
Negli ultimi decenni, lo studio dei sistemi complessi ha subito una profonda accelerazione grazie allo sviluppo della \emph{Network Science}, un framework teorico che permette di descrivere sistemi apparentemente disparati, dalle reti metaboliche nella cellula, alle infrastrutture tecnologiche come Internet, fino alle reti sociali, attraverso un linguaggio matematico unificato \cite[Cap. 1]{barabasi2016network}.
Una delle questioni centrali in questo ambito riguarda l'emergere spontaneo di proprietà macroscopiche a partire da interazioni locali disordinate. In particolare, è fondamentale comprendere i meccanismi che garantiscono la connettività globale del sistema e, di riflesso, la sua \textbf{robustezza} a fronte di fallimenti casuali dei componenti (\emph{random failures}).

Il presente elaborato si propone di analizzare la stabilità strutturale delle reti attraverso lo studio della \emph{teoria della percolazione}, applicata al modello di grafi casuali di Erdős-Rényi $G(N, p)$.
Come discussoda Newman \cite[Cap. 16]{newman2018networks}, la percolazione rappresenta una delle transizioni di fase geometriche più semplici e fondamentali: essa descrive il passaggio improvviso da uno stato frammentato, in cui il sistema è composto da isole disconnesse, a uno stato connesso dominato da una \emph{componente gigante} che abbraccia l'intero sistema.

Sebbene le reti reali presentino spesso correlazioni topologiche complesse (come la proprietà \emph{scale-free}), il modello di Erdős-Rényi-Gilbert costituisce il "modello nullo" di riferimento, analogo al gas ideale in termodinamica. In questo modello, la probabilità di connessione tra i nodi è uniforme e indipendente, permettendo di trattare la transizione di fase in approssimazione di campo medio (\emph{Mean Field}).
L'analisi di questo sistema possiede una forte valenza fisica: la soglia critica di percolazione identifica il punto esatto di rottura del sistema. Sotto questa soglia, la rete perde la sua funzionalità globale; sopra di essa, emerge l'ordine a lungo raggio.

\subsection{Obiettivi del lavoro}
Lo scopo centrale di questo elaborato è lo studio della \emph{teoria della percolazione} come chiave di lettura fondamentale per comprendere le transizioni di fase topologiche nelle reti complesse.
Utilizzando il modello di Erdős-Rényi-Gilbert come sistema di riferimento, il lavoro si prefigge i seguenti obiettivi:

\begin{itemize}
    \item \textbf{Analisi della soglia di percolazione:} Derivare analiticamente, mediante il formalismo delle funzioni generatrici \cite[Cap. 12]{newman2018networks}, la condizione critica $\langle k \rangle_c = 1$. Questo valore segna l'emersione del \emph{cluster di percolazione} (o componente gigante), corrispondente alla rottura di simmetria del sistema e alla nascita dell'ordine a lungo raggio.

    \item \textbf{Caratterizzazione dei fenomeni critici:} Studiare il comportamento della \emph{suscettività topologica} ($\chi$), intesa come la dimensione media dei cluster finiti. Attraverso simulazioni numeriche, si intende verificare la divergenza di $\chi$ in prossimità della soglia critica, evidenziando la natura continua (del secondo ordine) della transizione, in analogia con i sistemi termodinamici vicino al punto critico.

    \item \textbf{Implicazioni sulla robustezza:} Interpretare fisicamente la transizione di percolazione nel contesto della stabilità strutturale. Si discuterà come la scomparsa del cluster di percolazione (per $\langle k \rangle < 1$) corrisponda al punto di collasso funzionale (\emph{failure}) della rete, collegando così la meccanica statistica della percolazione al concetto di robustezza descritto da Barabási \cite[Cap. 8]{barabasi2016network}.
\end{itemize}

\section{Elementi di teoria dei grafi}
Prima di addentrarci nell'analisi del modello di Erdős-Rényi-Gilbert, è necessario introdurre alcuni concetti fondamentali della teoria dei grafi \cite[Sez. 2.3, 2.10]{barabasi2016network}. Un grafo $G$ è costituito da un insieme di nodi e un insieme di link che connettono coppie di nodi, rappresentando un'interazione tra di essi. La dimensione del grafo è data dal numero di nodi $N$, mentre la connettività è data dal numero totale di link $L$.
Una proprietà fondamentale di ciascun nodo è il suo grado $k$, che rappresenta il numero di link che esso possiede verso altri nodi.
In un grafo non orientato, il numero totale di link $L$, può essere espresso come \ext{la somma dei gradi dei nodi divisa per 2, poiché ogni link viene contato due volte (una volta per ciascuno dei due nodi che collega)}:
\begin{equation}
    L = \frac{1}{2} \sum_{i=1}^{N} k_i
\end{equation}
dove $k_i$ è il grado del nodo $i$. Il grado medio $\langle k \rangle$ è dato da:
\begin{equation}
    \langle k \rangle = \frac{1}{N} \sum_{i=1}^{N} k_i = \frac{2L}{N}
\end{equation}
La distribuzione dei gradi, $P(k)$, fornisce la probabilità che un nodo scelto casualmente nella rete abbia grado $k$. Poiché $P(k)$ è una probabilità, deve essere normalizzata, cioè:
\begin{equation}
    \sum_{k=1}^{\infty} P(k) = 1
\end{equation}
Per una rete con $N$ nodi, la distribuzione dei gradi è data da:
\begin{equation}
    P(k) = \frac{N_k}{N}
\end{equation}
dove $N_k$ è il numero di nodi di grado $k$.

Il coefficiente di clustering descrive il grado con cui i vicini di un dato nodo sono collegati tra loro. Per un nodo $i$ con grado $k_i$, il coefficiente di clustering locale è definito come:
\begin{equation}
    C_i = \frac{2 L_i}{k_i (k_i - 1)}
\end{equation}
dove $L_i$ rappresenta il numero di link tra i $k_i$ vicini del nodo $i$. Si noti che $C_i$ è compreso tra 0 e 1. \ext{Dunque,
$C_i$ misura la densità locale dei link nel network: più densamente interconnesso è il "vicinato" del nodo $i$, maggiore è il suo coefficiente di clustering locale.}
Il grado di clustering di un intero network è catturato dal coefficiente di clustering medio $\langle C \rangle$,definito come:
\begin{equation}
    \langle C \rangle = \frac{1}{N} \sum_{i=1}^{N} C_i
\end{equation}
In linea con l'interpretazione probabilistica, 〈C〉 è la probabilità che due vicini di un nodo scelto casualmente siano collegati tra loro.

\section{Percolazione nel Modello di Erdős-Rényi-Gilbert}
\section{Simulazione e risultati}
\section{Conclusioni}

\bibliographystyle{plain}
\bibliography{bibliografia}

\err{AGGIUSTA BIBLIO (pagine si/no?), ecc}

\end{document}